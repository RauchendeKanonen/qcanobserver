\documentclass[a4paper,10pt]{article}


%opening
\title{Howto QCANObserver}
\author{Netseal}

\begin{document}

\maketitle
\section{General}
First you need to install the Drivers for your CAN-Hardware. Currently there is only support for the PEAK-System interfaces. 
The USB-Interface is the only tested one but the others should work out of the box. If you have not got a CAN Interface, 
you can try the Program too, since it allows you to use all the functionality by loading sniffed information from a file. 
A little file is included. A little Database that shows the defition of Items too. 
The Software is the very first release and might steel your Beer. So pay attention!


\section{Driverinstallation}
Your linux kernel needs support for your CAN hardware and additinally support for the SocketCAN network interface.
Follow the instruction on the SocketCAN wiki site: http://en.wikipedia.org/wiki/SocketCAN.
Additinally see
After installation, plug in your hardware and load the drivers:
\newline
\begin{verbatim}
modprobe can_bcm
modprobe can_raw
modprobe can
modprobe pcan		//my PEAK interface
\end{verbatim}

lsmod should show the followin output:\newline
\begin{tabular}[h]{|p{2cm}|p{2cm}|p{3cm}|p{4cm}|}
\hline
\textbf{Module}&\textbf{Size}&\textbf{Used by}&\textbf{comment}\\
\hline
\hline
pcan & 28135 & 0 & my driver for the PEAK hardware\\
\hline
vcan & 28135 & 0 & virtual can interface driver (SocketCAN)\\
\hline
can raw & 4261 & 0& raw can interface driver (SocketCAN)\\
\hline
can bcm & 9366 & 0& bcm can interface driver (SocketCAN)\\
\hline
can & 19090 & 2 can bcm,can raw & driver (SocketCAN)\\
\hline
\end{tabular}\newline
\newline
- Create a virtual CAN network interface:
\begin{verbatim}     
ip link add type vcan
\end{verbatim}

I do not know if the can0/can1 ... networkinterfaces are created automatically on plug in. If not 
you need something like this for creation:
\begin{verbatim}     
ip link add can0
\end{verbatim} 

Establish a network interface with this little script:
\begin{verbatim}
 #/bin/bash
ifconfig can0 down
echo i 0x0014 e > /dev/pcanusb0			#for the peak driver 1 Mboud of datarate. 
#See the manual of your hardware
ifconfig can0 up
\end{verbatim} 


\section{Getting the Source}

The project is in a svn respository on Sourceforge.net.
You need a svn client to check it out. Look inside of your distribution Package Manager for this client.
With the following Code, you can check it out onto a local directory on your Computer:
\begin{verbatim}
 svn co https://qcanobserver.svn.sourceforge.net/svnroot/qcanobserver qcanobserver 
\end{verbatim} 

If you want to commit some changes, contact me on my sourceforge account. 

\section{Qt}
If you haven't, you should install the QT4 libraries. Also you need the Qwt libs. Best you try to install them from you Distro's package manager.
I found the QT-Creator useful. It is a lightwight IDE for GUI development.
Here one should be able to add specific ID's. Display the States of the Items inside the Message that you got out of the Database.
Follow your own ideas!

\section{Build QCANObserver}
You can build the Programm also without the Qt Creator. Execute qmake, then run make at the shell inside of the base directory.
Next you need to build the device library. Look inside of DeviceLib dir for your lib. Actually there is only an interface for the SocketCAN-network interface.
Run qmake and make inside. Then copy the .so files all into the lib dir of QCANObserver.





\section{Basic usage}
In the menu select a devicelib (the one you compiled and copied into) first. Click ok. Select the network interface on the next dialog in the pull down menu.
Or if it not appears, give the name in the field below.
Then select a database. In the menu select ``new'' -> Graphic. Add some Items to the Graphic window then. Start sniffing!

\subsection{Without a CAN-Interface}
In the menu select a database first. In the menu select ``new'' -> Graphic. Add some Items to the Graphic window then. Load a sniffed file!



\end{document}

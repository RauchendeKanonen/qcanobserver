\documentclass[a4paper,10pt]{article}


%opening
\title{Howto QCANObserver}
\author{Netseal}

\begin{document}

\maketitle
\section{General}
First you need to install the Drivers for your CAN-Hardware. Currently there is only support for the PEAK-System interfaces. The USB-Interface is the only tested one but the others should work out of the box. If you have not got a CAN Interface, you can try the Program too, since it allows you to use all the functionality by loading sniffed information from a file. A little file is included. A little Database that shows the defition of Items too. The Software is the very first release and might steel your Beer. So pay attention!


\section{Driverinstallation}
Go to www.peak-system.com and download the actual Drivers.
Open a shell, go to the dirctory where you downloaded the gzip file and execute something like this:
\begin{verbatim}
 tar xvf peak-linux-driver*.tar.gz
\end{verbatim} 
It is necessary that you meet the prequisites from the driverdocumentation.
At least there should be the Kernel-Headers of your running Kernel. Look in the documentation of your Linux-Distribution.\\
Build the Modul:
\begin{verbatim}
cd peak-linux-driver-x.y
make clean
make
su -c "make install"
\end{verbatim} 
If the compilation was successful, try to load the driver:

\begin{verbatim}
modprobe pcan
\end{verbatim} 
If no error occures:

\begin{verbatim}
lsmod | grep pcan
\end{verbatim} 
lsmod should show something like this:

\begin{verbatim}
pcan                   44412  0
\end{verbatim} 

\section{Getting the Source}

The project is in a svn respository on Sourceforge.net.
You need a svn client to check it out. Look inside of your distribution Package Manager for this client.
With the following Code, you can check it out onto a local directory on your Computer:
\begin{verbatim}
 svn co https://qcanobserver.svn.sourceforge.net/svnroot/qcanobserver qcanobserver 
\end{verbatim} 

If you want to commit some changes, contact me on my sourceforge account. 

\section{Qt}
If you haven't, you should install the QT4 libraries. Also you need the Qwt libs. Best you try to install them from you Distro's package manager.
I found the QT-Creator useful. It is a lightwight IDE for GUI development.
Open the .pro file inside of The Creator and add a Form-Class called Observer. It should be a dialog. 
Here one should be able to add specific ID's. Display the States of the Items inside the Message that you got out of the Database.
Follow your own ideas!
You can build the Programm also without the Qt Creator. Try make at the shell.

\section{Basic usage}
In the menu select a device first. Then select a database. In the menu select ``new'' -> Graphic. Add some Items to the Graphic window then. Start sniffing!

\subsection{Without a CAN-Interface}
In the menu select a database first. In the menu select ``new'' -> Graphic. Add some Items to the Graphic window then. Load a sniffed file!



\end{document}

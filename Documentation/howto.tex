\documentclass[a4paper,10pt]{article}


%opening
\title{Howto QCANObserver}
\author{Netseal}

\begin{document}

\maketitle
\section{General}
First you need to install the Drivers for your CAN-Hardware. Currently there is only support for the PEAK-System interfaces. The USB-Interface is the only tested one but the others should work out of the box. The Software is the very first release and might steel your Beer. So pay attention!


\section{Driverinstallation}
Go to www.peak-system.com and download the actual Drivers.
Open a shell, go to the dirctory where you downloaded the gzip file and execute something like this:
\begin{verbatim}
 tar xvf peak-linux-driver*.tar.gz
\end{verbatim} 
It is necessary that you meet the prequisites from the driverdocumentation.
At least there should be the Kernel-Headers of your running Kernel. Look in the documentation of your Linux-Distribution.\\
Build the Modul:
\begin{verbatim}
cd peak-linux-driver-x.y
make clean
make
su -c "make install"
\end{verbatim} 
If the compilation was successful, try to load the driver:

\begin{verbatim}
modprobe pcan
\end{verbatim} 
If no error occures:

\begin{verbatim}
lsmod | grep pcan
\end{verbatim} 
lsmod should show something like this:

\begin{verbatim}
pcan                   44412  0
\end{verbatim} 

\section{QT}
If you haven't, you should install the QT4 libraries. Also you need the Qwt libs. Best you try to install them from you Distro's package manager.
I found the QT-Creator useful. It is a lightwight IDE for GUI development.
Open the .pro file inside of The Creator and add a Form-Class called Observer. It should be a dialog. 
Here one should be able to add specific ID's. Display the States of the Items inside the Message that you got out of the Database.
Follow your own ideas!

\end{document}
